\documentclass{article}
\usepackage{amsmath,amssymb,listings,upquote}
\usepackage[margin=3cm]{geometry}
\usepackage{graphicx,color}
\lstset{language=Python}
\usepackage{enumerate}

\newcounter{zone}
\setcounter{zone}{0}
\newcommand{\zone}{\clearpage\refstepcounter{zone}\section*{Zone \arabic{zone}}}
\newcounter{question}
\setcounter{question}{0}
\newcounter{variant}
\newcounter{questionpoints}
\newcommand{\question}[1]{\newpage \refstepcounter{question} \setcounter{variant}{0} \setcounter{questionpoints}{#1}}
\newcommand{\variant}{\vspace{4em}\refstepcounter{variant}\noindent \arabic{question}/\arabic{variant}. (\arabic{questionpoints} point\ifnum \thequestionpoints > 1 s\fi) }
\newenvironment{answers}{\begin{enumerate}}{\end{enumerate}}
\newcommand{\answer}{\item }
\newcommand{\correctanswer}{\item $\bigstar$ }
\renewcommand{\theenumi}{\Alph{enumi}}
\newenvironment{solution}{{\bf Solution.} }{\vspace*{.3in}\hrule}

\begin{document}

\begin{center}
\textbf{\Large CS101 Practice Midterm 1}
\end{center}

\bigskip
\noindent
\begin{itemize}
\item \textbf{Be sure to enter your \underline{NetID} and \underline{the code below} on your Scantron}.
\item Do not turn this page until instructed to.
\item There are 25 questions worth 1 point each.
\item Each question has only \textbf{one} correct answer.
\item You must not communicate with other students during this test.
\item No books, notes, or electronic devices allowed.
\item This is a 45 minute exam.
\item There are several different versions of this exam.
\end{itemize}

\bigskip\bigskip
\noindent
\textbf{\Large 1. Fill in your information:}

\bigskip
{\Large\bf
\begin{tabular}{ll}
Full Name: & \underbar{\hskip 8cm} \\[0.5em]
UIN (Student Number): & \underbar{\hskip 8cm} \\[0.5em]
NetID: & \underbar{\hskip 8cm}
\end{tabular}
}

\bigskip
\bigskip
\noindent
\textbf{\Large 2. Fill in the following answers on the Scantron form:}

%%%%%%%%%%%%%%%%%%%%%%%%%%%%%%%%%%%%%%%%%%%%%%%%%%%%%%%%%%%%%%%%%%%%%%
%%%%%%%%%%%%%%%%%%%%%%%%%%%%%%%%%%%%%%%%%%%%%%%%%%%%%%%%%%%%%%%%%%%%%%
\zone

%%%%%%%%%%%%%%%%%%%%%%%%%%%%%%%%%%%%%%%%%
\question{1}
\variant
Which of the following represents a single valid string?
\begin{answers}
\answer \begin{verbatim}''What's your business here?' he demanded, grimly. 'Who are you?''\end{verbatim}
\answer \begin{verbatim}'"I'll keep him out five minutes," he exclaimed. "You won't object?"'\end{verbatim}
\answer \begin{verbatim}'"I'll not hold my tongue!" I said. "Let the door remain shut, and be quiet!"'\end{verbatim}
\correctanswer \begin{verbatim}"'What has Heathcliff done to you?' I asked. 'In what has he wronged you?'"\end{verbatim}
\answer None of the other answers form a single valid string.
\end{answers}
\begin{solution}
\end{solution}

%%%%%%%%%%%%%%%%%%%%%%%%%%%%%%%%%%%%%%%%%

\question{1}
\variant
Evaluating which of the following expressions will produce a value of type list?
\begin{answers}
\answer  \begin{verbatim}len([3333])\end{verbatim}
\correctanswer  \begin{verbatim}["1","2","3"]+["4"] \end{verbatim}
\answer  \begin{verbatim}list("ABC").append("D")\end{verbatim}
\answer  \begin{verbatim}str(["A","B"]).lower() \end{verbatim}
\end{answers}
\begin{solution}
\end{solution}

%%%%%%%%%%%%%%%%%%%%%%%%%%%%%%%%%%%%%%%%%
\question{1}
\variant
Consider the following program:
\begin{verbatim}
x=["tick","tock"]
x[0]=x.sort()
x=x[-2]
\end{verbatim}
What is the \textbf{type} of x after the program is run?
\begin{answers}
\correctanswer NoneType (value \texttt{None})
\answer Tuple
\answer List
\answer String
\answer None of the other answers are correct.
\end{answers}
\begin{solution}
\end{solution}
%%%%%%%%%%%%%%%%%%%%%%%%%%%%%%%%%%%%%%%%%
\question{1}
\variant
Consider the following program:
\begin{verbatim}
x=["tick","tock"]
x[0]=x.reverse()
x=x[-2]
\end{verbatim}
What is the \textbf{type} of x after the program is run?
\begin{answers}
\correctanswer NoneType (value \texttt{None})
\answer Tuple
\answer List
\answer String
\answer None of the other answers are correct.
\end{answers}
\begin{solution}
\end{solution}

%%%%%%%%%%%%%%%%%%%%%%%%%%%%%%%%%%%%%%%%%
\question{1}
\variant
Consider the following program:
\begin{verbatim}
x=["tick","tock"]
x[-1]=list(x[0])
x=x[1],x[0]
\end{verbatim}
What is the \textbf{type} of x after the program is run?
\begin{answers}
\answer NoneType (value \texttt{None})
\correctanswer Tuple
\answer List
\answer String
\answer None of the other answers are correct.
\end{answers}
\begin{solution}
\end{solution}
%%%%%%%%%%%%%%%%%%%%%%%%%%%%%%%%%%%%%%%%%
\question{1}
\variant
Consider the following program:
\begin{verbatim}
x=["tick","tock"]
x[0]=len(list(x[-1]))
x=x[-2]
\end{verbatim}
What is the \textbf{type} of x after the program is run?
\begin{answers}
\answer NoneType (value \texttt{None})
\correctanswer Integer
\answer List
\answer String
\answer None of the other answers are correct.
\end{answers}
\begin{solution}
\end{solution}
%%%%%%%%%%%%%%%%%%%%%%%%%%%%%%%%%%%%%%%%%
\question{1}
\variant
Consider the following program:
\begin{verbatim}
x=["tick","tock"]
x[0]=(len(list(x[-1])),x[1])
x=x[1]
\end{verbatim}
What is the \textbf{type} of x after the program is run?
\begin{answers}
\answer NoneType (value \texttt{None})
\answer Integer
\answer List
\correctanswer String
\answer None of the other answers are correct.
\end{answers}
\begin{solution}
\end{solution}
%%%%%%%%%%%%%%%%%%%%%%%%%%%%%%%%%%%%%%%%%
\question{1}
\variant
Consider the following program:
\begin{verbatim}
a=list("REDUCIO")
a.sort()
a[0],a[1]=a[-2],a[-1]
x=""
for e in a:
    x=x+e
\end{verbatim}
What is the \textbf{value} of x after this program is executed?
\begin{answers}
\correctanswer \begin{verbatim}"RUEIORU"\end{verbatim}
\answer \begin{verbatim}"IODUCIO"\end{verbatim}
\answer \begin{verbatim}"UREIORU"\end{verbatim}
\answer \begin{verbatim}"OIDUCIO"\end{verbatim}
\answer None of the other answers are correct.
\end{answers}
\begin{solution}
\end{solution}

%%%%%%%%%%%%%%%%%%%%%%%%%%%%%%%%%%%%%%%%%
\question{1}
\variant
Consider the following program:
\begin{verbatim}
a=list("ACCIO")
a.reverse()
a[1],a[2]=a[2],a[3]
x=""
for e in a:
    x=x+e
\end{verbatim}
What is the \textbf{value} of x after this program is executed?
\begin{answers}
\correctanswer \begin{verbatim}"OCCCA"\end{verbatim}
\answer \begin{verbatim}"ACCCO"\end{verbatim}
\answer \begin{verbatim}"OIICC"\end{verbatim}
\answer \begin{verbatim}"AIICC"\end{verbatim}
\answer None of the other answers are correct.
\end{answers}
\begin{solution}
\end{solution}

%%%%%%%%%%%%%%%%%%%%%%%%%%%%%%%%%%%%%%%%%
\question{1}
\variant
Consider the following program:
\begin{verbatim}
s="CHARIZARD"
x=""
for i in range(0,len(s)):
    if (i>3) and (i<6):
        x+=s[i:i+2]
\end{verbatim}
What is the \textbf{value} of x after this program is executed?
\begin{answers}
\correctanswer \begin{verbatim}"IZZA"\end{verbatim}
\answer \begin{verbatim}"ZA"\end{verbatim}
\answer \begin{verbatim}"RI"\end{verbatim}
\answer \begin{verbatim}"RIIZ"\end{verbatim}
\answer None of the other answers are correct.
\end{answers}
\begin{solution}
\end{solution}

%%%%%%%%%%%%%%%%%%%%%%%%%%%%%%%%%%%%%%%%%
\question{1}
\variant
Consider the following program:
\begin{verbatim}
def fun(a,b):
    if a>b and a!=4:
        return b==5
    else:
        return a==3
a=5
b=4
print(fun(a,b))
\end{verbatim}
What is printed out by this program?
\begin{answers}
\correctanswer False
\answer 4
\answer 5
\answer True
\answer None of the other answers. This code is not valid.
\end{answers}
\begin{solution}
\end{solution}

%%%%%%%%%%%%%%%%%%%%%%%%%%%%%%%%%%%%%%%%%
\question{1}
\variant
Consider the following incomplete function.
\begin{verbatim}
def pal(s):
    a=list(s)
    if ???:
        return True
    else:
        return False
\end{verbatim}
The function is intended to return True if and only if the input string s is a palindrome. A palindrome is a string that reads the same forward and backward, like ``ABBA'' or ``RACECAR''. What should replace the three question marks to complete the function?
\begin{answers}
\answer  \begin{verbatim} a + a == a * 2 \end{verbatim}
\answer  \begin{verbatim} a.reverse()==a \end{verbatim}
\answer  \begin{verbatim}(len(a) % 2) == 0 \end{verbatim}
\correctanswer  None of the other answers are correct.
\end{answers}
\begin{solution}
\end{solution}

%%%%%%%%%%%%%%%%%%%%%%%%%%%%%%%%%%%%%%%%%
\question{1}
\variant
Consider the following program:
\begin{verbatim}
s="SQUIRTLE"
x=""
for i in range(0,len(s)):
    if (i>4) and (i<7):
        x+=s[i:i+2]
\end{verbatim}
What is the \textbf{value} of x after this program is executed?
\begin{answers}
\correctanswer \begin{verbatim}"TLLE"\end{verbatim}
\answer \begin{verbatim}"RTTL"\end{verbatim}
\answer \begin{verbatim}"RTTLLE"\end{verbatim}
\answer \begin{verbatim}"RT"\end{verbatim}
\answer None of the other answers are correct.
\end{answers}
\begin{solution}
\end{solution}
%%%%%%%%%%%%%%%%%%%%%%%%%%%%%%%%%%%%%%%%%
\question{1}
\variant
Consider the following program:
\begin{verbatim}
s="MEWTWO"
x=""
for i in range(0,len(s)):
    if (i>1) and (i<3):
        x+=s[i:i+3]

\end{verbatim}
What is the \textbf{value} of x after this program is executed?
\begin{answers}
\correctanswer \begin{verbatim}"WTW"\end{verbatim}
\answer \begin{verbatim}"EWT"\end{verbatim}
\answer \begin{verbatim}"WTWO"\end{verbatim}
\answer \begin{verbatim}"EWTW"\end{verbatim}
\answer None of the other answers are correct.
\end{answers}
\begin{solution}
\end{solution}

%%%%%%%%%%%%%%%%%%%%%%%%%%%%%%%%%%%%%%%%%
\question{1}
\variant
Consider the following program.
\begin{verbatim}
def fun(a,b):
    return a-b
x=0
for i in range(1,4):
    x=x+fun(i,x)
\end{verbatim}
After it is run, what is the final \textbf{value} of x?
\begin{answers}
\correctanswer \begin{verbatim}3\end{verbatim}
\answer \begin{verbatim}4\end{verbatim}
\answer \begin{verbatim}5\end{verbatim}
\answer None of the other answers are correct.
\end{answers}
\begin{solution}
\end{solution}
%%%%%%%%%%%%%%%%%%%%%%%%%%%%%%%%%%%%%%%%%
\question{1}
\variant
Consider the following program.
\begin{verbatim}
def fun(a,b):
    return a-b
x=0
for i in range(2,5):
    x=x+fun(i,x)
    print(x)
\end{verbatim}
After it is run, what is the final \textbf{value} of x?
\begin{answers}
\answer \begin{verbatim}3\end{verbatim}
\correctanswer \begin{verbatim}4\end{verbatim}
\answer \begin{verbatim}5\end{verbatim}
\answer None of the other answers are correct.
\end{answers}
\begin{solution}
\end{solution}
%%%%%%%%%%%%%%%%%%%%%%%%%%%%%%%%%%%%%%%%%
\question{1}
\variant
Consider the following program.
\begin{verbatim}
def fun(a,b):
    return a-b
x=0
for i in range(-1,3):
    x=x+fun(i,x)
    print(x)
\end{verbatim}
After it is run, what is the final \textbf{value} of x?
\begin{answers}
\correctanswer \begin{verbatim}2\end{verbatim}
\answer \begin{verbatim}3\end{verbatim}
\answer \begin{verbatim}4\end{verbatim}
\answer None of the other answers are correct.
\end{answers}
\begin{solution}
\end{solution}

%%%%%%%%%%%%%%%%%%%%%%%%%%%%%%%%%%%%%%%%%
\question{1}
\variant
Evaluate the following expression:
\begin{verbatim}
len("ABCD"[1:3])
\end{verbatim}
What value is produced?
\begin{answers}
\answer 1
\answer 3
\correctanswer 2
\answer 4
\end{answers}
\begin{solution}
\end{solution}

%%%%%%%%%%%%%%%%%%%%%%%%%%%%%%%%%%%%%%%%%
\question{1}
\variant
Consider the following program.
\begin{verbatim}
x=0
i=1
while(i*i)<=36:
    if ((i*i)%2)==0:
        x+=1
    i=i+1
\end{verbatim}
After it is run, what is the final \textbf{value} of x?
\begin{answers}
\correctanswer \begin{verbatim}3\end{verbatim}
\answer \begin{verbatim}4\end{verbatim}
\answer \begin{verbatim}5\end{verbatim}
\answer None of the other answers are correct.
\end{answers}
\begin{solution}
\end{solution}
%%%%%%%%%%%%%%%%%%%%%%%%%%%%%%%%%%%%%%%%%
\question{1}
\variant
Consider the following program.
\begin{verbatim}
x=0
i=1
while(i*i)<=49:
    if (i%2)==1:
        x+=1
    i=i+1
\end{verbatim}
After it is run, what is the final \textbf{value} of x?
\begin{answers}
\answer \begin{verbatim}3\end{verbatim}
\correctanswer \begin{verbatim}4\end{verbatim}
\answer \begin{verbatim}5\end{verbatim}
\answer None of the other answers are correct.
\end{answers}
\begin{solution}
\end{solution}

%%%%%%%%%%%%%%%%%%%%%%%%%%%%%%%%%%%%%%%%%
\question{1}
\variant
Evaluate the following expression:
\begin{verbatim}
"ABC".join(["A","B","C"])
\end{verbatim}
What value is produced?
\begin{answers}
\answer \begin{verbatim}"ABCABCABC"\end{verbatim}
\correctanswer \begin{verbatim}"AABCBABCC"\end{verbatim}
\answer \begin{verbatim}"AAABBBCCC"\end{verbatim}
\answer None of the other answers are correct.
\end{answers}
\begin{solution}
\end{solution}

%%%%%%%%%%%%%%%%%%%%%%%%%%%%%%%%%%%%%%%%%
\question{1}
\variant
Evaluate the following expression:
\begin{verbatim}
"+".join("ABABABA".split("A"))
\end{verbatim}
What value is produced?
\begin{answers}
\answer \begin{verbatim}"ABABABA"\end{verbatim}
\correctanswer \begin{verbatim}"+B+B+B+"\end{verbatim}
\answer \begin{verbatim}"B+B+B"\end{verbatim}
\answer None of the other answers are correct.
\end{answers}
\begin{solution}
\end{solution}

%%%%%%%%%%%%%%%%%%%%%%%%%%%%%%%%%%%%%%%%%
\question{1}
\variant
Consider the following program:
\begin{verbatim}
def fun(a,b):
     for i in range(a,b):
        if (i%3)==0:
            return i
     return a==b

a=4
b=6
print(fun(a,b))
\end{verbatim}
What is printed out by this program?
\begin{answers}
\correctanswer False
\answer 6
\answer 3
\answer True
\answer None of the other answers. This code is not valid.
\end{answers}
\begin{solution}
\end{solution}

%%%%%%%%%%%%%%%%%%%%%%%%%%%%%%%%%%%%%%%%%
\question{1}
\variant
Consider the following program:
\begin{verbatim}
s="GABE&TYCHO"
x=s[3:6]
\end{verbatim}
What is the \textbf{value} of x after this program is executed?
\begin{answers}
\correctanswer \begin{verbatim}"E&T"\end{verbatim}
\answer \begin{verbatim}"E&"\end{verbatim}
\answer \begin{verbatim}"BE&"\end{verbatim}
\answer \begin{verbatim}"BE"\end{verbatim}
\answer None of the other answers are correct.
\end{answers}
\begin{solution}
\end{solution}

%%%%%%%%%%%%%%%%%%%%%%%%%%%%%%%%%%%%%%%%%
\question{1}
\variant
Consider the following program:
\begin{verbatim}
s="A,E,I,O,U".split(",")
s=s[0:3]
s=s.sort()
\end{verbatim}
What is the \textbf{value} of s after this program is executed?
\begin{answers}
\answer \begin{verbatim}['A', 'E', 'I']\end{verbatim}
\answer \begin{verbatim}['A', 'E', 'I','O']\end{verbatim}
\answer \begin{verbatim}"AEI"\end{verbatim}
\answer \begin{verbatim}"AEIO"\end{verbatim}
\correctanswer None of the other answers are correct.
\end{answers}
\begin{solution}
\end{solution}

%%%%%%%%%%%%%%%%%%%%%%%%%%%%%%%%%%%%%%%%%
\question{1}
\variant
For this problem, you should compose a function which accomplishes a given task using the available code blocks arranged in the correct functional order.  \emph{We ignore indentation for this problem.}

\texttt{find\_min} should accept a \texttt{list} and return the value of the \emph{minimum item} in the \texttt{list}.  (We use a large value to initialize our comparison in \texttt{min\_val}.)

\begin{verbatim}
def find_min(my_list):
\end{verbatim}

\begin{enumerate}[1]
\item \texttt{min\_val = i}
\item \texttt{min\_val = 1e300}
\item \texttt{for i in range(len(my\_list)):}
\item \texttt{if i < min\_val:}
\item \texttt{min\_val = my\_list[i]}
\item \texttt{return min\_val}
\item \texttt{if my\_list[i] < min\_val:}
\item \texttt{for i in range(my\_list):}
\item \texttt{print(min\_val)}
\end{enumerate}

\begin{answers}
\answer  2, 3, 4, 1, 6
\answer  3, 2, 7, 5, 9
\correctanswer  2, 3, 7, 5, 6
\answer  2, 8, 4, 5, 6
\answer  2, 3, 7, 1, 6
\end{answers}
\begin{solution}
\end{solution}

\end{document}
