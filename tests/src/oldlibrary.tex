\documentclass{article}
\usepackage{amsmath,amssymb,listings,upquote}
\usepackage[margin=3cm]{geometry}
\usepackage{graphicx,color}
\lstset{language=Python}


\newcounter{zone}
\setcounter{zone}{0}
\newcommand{\zone}{\clearpage\refstepcounter{zone}\section*{Zone \arabic{zone}}}
\newcounter{question}
\setcounter{question}{0}
\newcounter{variant}
\newcounter{questionpoints}
\newcommand{\question}[1]{\newpage \refstepcounter{question} \setcounter{variant}{0} \setcounter{questionpoints}{#1}}
\newcommand{\variant}{\vspace{4em}\refstepcounter{variant}\noindent \arabic{question}/\arabic{variant}. (\arabic{questionpoints} point\ifnum \thequestionpoints > 1 s\fi) }
\newenvironment{answers}{\begin{enumerate}}{\end{enumerate}}
\newcommand{\answer}{\item }
\newcommand{\correctanswer}{\item $\bigstar$ }
\renewcommand{\theenumi}{\Alph{enumi}}
\newenvironment{solution}{{\bf Solution.} }{\vspace*{.3in}\hrule}

\begin{document}

\begin{center}
\textbf{\Large CS101 Midterm 1}
\end{center}

\bigskip
\noindent
\begin{itemize}
\item \textbf{Be sure to enter your \underline{NetID} and \underline{the code below} on your Scantron}.
\item Do not turn this page until instructed to.
\item There are 25 questions worth 1 point each.
\item Each question has only \textbf{one} correct answer.
\item You must not communicate with other students during this test.
\item No books, notes, or electronic devices allowed.
\item This is a 45 minute exam.
\item There are several different versions of this exam.
\end{itemize}

\bigskip\bigskip
\noindent
\textbf{\Large 1. Fill in your information:}

\bigskip
{\Large\bf
\begin{tabular}{ll}
Full Name: & \underbar{\hskip 8cm} \\[0.5em]
UIN (Student Number): & \underbar{\hskip 8cm} \\[0.5em]
NetID: & \underbar{\hskip 8cm}
\end{tabular}
}

\bigskip
\bigskip
\noindent
\textbf{\Large 2. Fill in the following answers on the Scantron form:}

%%%%%%%%%%%%%%%%%%%%%%%%%%%%%%%%%%%%%%%%%%%%%%%%%%%%%%%%%%%%%%%%%%%%%%
%%%%%%%%%%%%%%%%%%%%%%%%%%%%%%%%%%%%%%%%%%%%%%%%%%%%%%%%%%%%%%%%%%%%%%
\zone

%%%%%%%%%%%%%%%%%%%%%%%%%%%%%%%%%%%%%%%%%
\question{1}
\variant
What do we call a set of instructions a computer executes to achieve a goal?
\begin{answers}
\correctanswer a program
\answer a process
\answer a prologue 
\answer a protest
\answer None of the other answers are correct.
\end{answers}
\begin{solution}
\end{solution}
\variant
What do we call information stored in a computer (usually represented in binary?)
\begin{answers}
\correctanswer data
\answer date
\answer dave
\answer dance
\answer None of the other answers are correct.
\end{answers}
\begin{solution}
\end{solution}

%%%%%%%%%%%%%%%%%%%%%%%%%%%%%%%%%%%%%%%%%
\question{1}
\variant
What do we call a grouping of 8 bits?
\begin{answers}
\correctanswer a byte
\answer a bill
\answer a board
\answer a brush
\answer None of the other answers are correct.
\end{answers}
\begin{solution}
\end{solution}
\variant
How many bits are grouped together to form a byte?
\begin{answers}
\correctanswer 8
\answer 16
\answer 4
\answer 2
\answer None of the other answers are correct.
\end{answers}
\begin{solution}
\end{solution}

%%%%%%%%%%%%%%%%%%%%%%%%%%%%%%%%%%%%%%%%%
\question{1}
\variant
Which of the following statements is true of Python?
\begin{answers}
\answer It is a high level language.
\answer It is an interpreted language.
\answer It has both syntax and semantics.
\correctanswer All of the other answers are correct.
\end{answers}
\begin{solution}
\end{solution}
\variant
Which of the following statements is \emph{not} true of Python?
\begin{answers}
\answer It is a low level language.
\answer It has syntax, but not semantics.
\answer It has semantics, but not syntax.
\correctanswer All of the other answers are correct.
\end{answers}
\begin{solution}
\end{solution}
\variant
Which of the following statements is true of Python?
\begin{answers}
\answer It is a low level language.
\correctanswer It is an interpreted language.
\answer It has semantics, but not syntax.
\answer All of the other answers are correct.
\end{answers}
\begin{solution}
\end{solution}
%%%%%%%%%%%%%%%%%%%%%%%%%%%%%%%%%%%%%%%%%
\question{1}
\variant
Consider the following program:
\begin{verbatim}
s="KVOTHE"
x=s[1:3]
\end{verbatim}
What is the \textbf{value} of x after this program is executed?
\begin{answers}
\answer \begin{verbatim}"KV"\end{verbatim}
\answer \begin{verbatim}"KVO"\end{verbatim}
\correctanswer \begin{verbatim}"VO"\end{verbatim}
\answer \begin{verbatim}"VOT"\end{verbatim}
\answer None of the other answers are correct.
\end{answers}
\begin{solution}
\end{solution}
\variant
Consider the following program:
\begin{verbatim}
s="DANNY RAND"
x=s[5:-1]
\end{verbatim}
What is the \textbf{value} of x after this program is executed?
\begin{answers}
\correctanswer \begin{verbatim}" RAN"\end{verbatim}
\answer \begin{verbatim}"RAND"\end{verbatim}
\answer \begin{verbatim}"RAN"\end{verbatim}
\answer \begin{verbatim}" RAND"\end{verbatim}
\answer None of the other answers are correct.
\end{answers}
\begin{solution}
\end{solution}
\variant
Consider the following program:
\begin{verbatim}
s="ARYA STARK"
x=s[3:5]
\end{verbatim}
What is the \textbf{value} of x after this program is executed?
\begin{answers}
\correctanswer \begin{verbatim}"A "\end{verbatim}
\answer \begin{verbatim}"YA"\end{verbatim}
\answer \begin{verbatim}"AS"\end{verbatim}
\answer \begin{verbatim}"YA S"\end{verbatim}
\answer None of the other answers are correct.
\end{answers}
\begin{solution}
\end{solution}

%%%%%%%%%%%%%%%%%%%%%%%%%%%%%%%%%%%%%%%%%
\question{1}
\variant
Consider the following program:
\begin{verbatim}
a=1
def fun():
    a=2
    return a+1
fun()
print a
\end{verbatim}
What is printed out by this program?
\begin{answers}
\correctanswer 1
\answer 2
\answer 3
\answer 4
\answer None of the other answers. This code is not valid.
\end{answers}
\begin{solution}
\end{solution}
\variant
Consider the following program:
\begin{verbatim}
a=1
def fun(a):
    a=2
    return a+1
    a=4
a=fun(a)
print a
\end{verbatim}
What is printed out by this program?
\begin{answers}
\answer 1
\answer 2
\correctanswer 3
\answer 4
\answer None of the other answers. This code is not valid.
\end{answers}
\begin{solution}
\end{solution}
%%%%%%%%%%%%%%%%%%%%%%%%%%%%%%%%%%%%%%%%%
\question{1}
\variant
Consider the following program:
\begin{verbatim}
def fun(a,b):
    if a<b and a==4:
        return a
    else:
        return b
a=5
b=4
print fun(b,a)
\end{verbatim}
What is printed out by this program?
\begin{answers}
\answer False
\correctanswer 4
\answer 5
\answer True
\answer None of the other answers. This code is not valid.
\end{answers}
\begin{solution}
\end{solution}
\variant
Consider the following program:
\begin{verbatim}
def fun(a,b):
    if a<b or a==5:
        return a
    else:
        return b
a=5
b=4
print fun(b,a)
\end{verbatim}
What is printed out by this program?
\begin{answers}
\answer False
\correctanswer 4
\answer 5
\answer True
\answer None of the other answers. This code is not valid.
\end{answers}
\begin{solution}
\end{solution}

%%%%%%%%%%%%%%%%%%%%%%%%%%%%%%%%%%%%%%%%%
\question{1}
\variant
Evaluate the following expression:
\begin{verbatim}
(True and 3!=4) or not (3<5 and True)
\end{verbatim}
What value is produced?
\begin{answers}
\answer False
\correctanswer True
\end{answers}
\begin{solution}
\end{solution}
\variant
Evaluate the following expression:
\begin{verbatim}
(True or 3==4) or ((not 3>5) and True)
\end{verbatim}
What value is produced?
\begin{answers}
\answer False
\correctanswer True
\end{answers}
\begin{solution}
\end{solution}
\variant
Evaluate the following expression:
\begin{verbatim}
(False or 3==4) and (False != True)
\end{verbatim}
What value is produced?
\begin{answers}
\correctanswer False
\answer True
\end{answers}
\begin{solution}
\end{solution}

%%%%%%%%%%%%%%%%%%%%%%%%%%%%%%%%%%%%%%%%%
\question{1}
\variant
Consider the following program.
\begin{verbatim}
x=False
for i in range(0,3):
    x=x and not x
\end{verbatim}
After it is run, what is the final \textbf{value} of x?
\begin{answers}
\correctanswer False
\answer True
\end{answers}
\begin{solution}
\end{solution}
\variant
Consider the following program.
\begin{verbatim}
y=1
for i in range(0,3):
    y=y+(2*y)
x=y>10
\end{verbatim}
After it is run, what is the final \textbf{value} of x?
\begin{answers}
\answer False
\correctanswer True
\end{answers}
\begin{solution}
\end{solution}
%%%%%%%%%%%%%%%%%%%%%%%%%%%%%%%%%%%%%%%%%
\question{1}
\variant
Consider the following program.
\begin{verbatim}
x=[]
for j in range(0,4):
    if (j%3)==0:
        x.append("*")
\end{verbatim}
After it is run, what is the final \textbf{value} of x?
\begin{answers}
\correctanswer \begin{verbatim}["*","*"]\end{verbatim}
\answer \begin{verbatim}["*","*","*"]\end{verbatim}
\answer \begin{verbatim}["*"]\end{verbatim}
\answer None of the other answers are correct.
\end{answers}
\begin{solution}
\end{solution}
\variant
Consider the following program.
\begin{verbatim}
x=[]
for j in range(2,7):
    if (j%2)==0:
        x.append("*")
\end{verbatim}
After it is run, what is the final \textbf{value} of x?
\begin{answers}
\answer \begin{verbatim}["*","*"]\end{verbatim}
\correctanswer \begin{verbatim}["*","*","*"]\end{verbatim}
\answer \begin{verbatim}["*"]\end{verbatim}
\answer None of the other answers are correct.
\end{answers}
\begin{solution}
\end{solution}
%%%%%%%%%%%%%%%%%%%%%%%%%%%%%%%%%%%%%%%%%
\question{1}
\variant
Consider the following program.
\begin{verbatim}
x=0
i=1
while(i*i)<=5:
    x=x+(i*i)
    i=i+1
\end{verbatim}
After it is run, what is the final \textbf{value} of x?
\begin{answers}
\correctanswer \begin{verbatim}5\end{verbatim}
\answer \begin{verbatim}10\end{verbatim}
\answer \begin{verbatim}13\end{verbatim}
\answer None of the other answers are correct.
\end{answers}
\begin{solution}
\end{solution}
\variant
Consider the following program.
\begin{verbatim}
x=0
i=1
while(i*i)<9:
    x=x+(i*i)
    i=i+1
\end{verbatim}
After it is run, what is the final \textbf{value} of x?
\begin{answers}
\correctanswer \begin{verbatim}5\end{verbatim}
\answer \begin{verbatim}10\end{verbatim}
\answer \begin{verbatim}13\end{verbatim}
\answer None of the other answers are correct.
\end{answers}
\begin{solution}
\end{solution}
\variant
Consider the following program.
\begin{verbatim}
x=0
i=1
while(i*i)<8:
    x=x+(i*i)
    i=i+1
\end{verbatim}
After it is run, what is the final \textbf{value} of x?
\begin{answers}
\correctanswer \begin{verbatim}5\end{verbatim}
\answer \begin{verbatim}10\end{verbatim}
\answer \begin{verbatim}13\end{verbatim}
\answer None of the other answers are correct.
\end{answers}
\begin{solution}
\end{solution}



%%%%%%%%%%%%%%%%%%%%%%%%%%%%%%%%%%%%%%%%%
\question{1}
\variant
Consider the following program.
\begin{verbatim}
x=0
n=[3,1,2,4]
n.append(0)
n.sort()
if n[0]==1:
    del n[1]
elif n[0]==0:
    del n[0]
else:
    del n[-1]
for i in n:
    x=x+i
\end{verbatim}
After it is run, what is the final \textbf{value} of x?
\begin{answers}
\correctanswer \begin{verbatim}10\end{verbatim}
\answer \begin{verbatim}9\end{verbatim}
\answer \begin{verbatim}7\end{verbatim}
\answer \begin{verbatim}6\end{verbatim}
\end{answers}
\begin{solution}
\end{solution}
\variant
Consider the following program.
\begin{verbatim}
x=0
n=[1,3,5]
n.append(4)
n.sort()
if n[0]==4:
    del n[1]
elif n[2]==3:
    del n[0]
else:
    del n[-1]
for i in n:
    x=x+i
\end{verbatim}
After it is run, what is the final \textbf{value} of x?
\begin{answers}
\answer \begin{verbatim}10\end{verbatim}
\correctanswer \begin{verbatim}8\end{verbatim}
\answer \begin{verbatim}12\end{verbatim}
\answer \begin{verbatim}9\end{verbatim}
\answer None of the other answers are correct.
\end{answers}
\begin{solution}
\end{solution}

%%%%%%%%%%%%%%%%%%%%%%%%%%%%%%%%%%%%%%%%%
\question{1}
\variant
Consider the following program:
\begin{verbatim}
x=3
a=5
if (a%3)==2:
    x=x**3
elif(a%3)==1:
    x=x**2
else:
    x=x**1
\end{verbatim}
What is the \textbf{value} of x after this program is executed?
\begin{answers}
\answer \begin{verbatim}3\end{verbatim}
\answer \begin{verbatim}9\end{verbatim}
\correctanswer \begin{verbatim}27\end{verbatim}
\answer \begin{verbatim}1\end{verbatim}
\answer None of the other answers are correct.
\end{answers}
\begin{solution}
\end{solution}
\variant
Consider the following program:
\begin{verbatim}
x=2
a=6
if (a%3)==2:
    x=x**3
elif(a%3)==1:
    x=x**2
else:
    x=x**1
\end{verbatim}
What is the \textbf{value} of x after this program is executed?
\begin{answers}
\answer \begin{verbatim}4\end{verbatim}
\answer \begin{verbatim}16\end{verbatim}
\answer \begin{verbatim}8\end{verbatim}
\correctanswer \begin{verbatim}2\end{verbatim}
\answer None of the other answers are correct.
\end{answers}
\begin{solution}
\end{solution}
\variant
Consider the following program:
\begin{verbatim}
x=3
a=7
if (a%3)==2:
    x=x**2
elif(a%3)==1:
    x=x**1
else:
    x=x**0
\end{verbatim}
What is the \textbf{value} of x after this program is executed?
\begin{answers}
\answer \begin{verbatim}1\end{verbatim}
\answer \begin{verbatim}9\end{verbatim}
\answer \begin{verbatim}7\end{verbatim}
\correctanswer \begin{verbatim}3\end{verbatim}
\answer None of the other answers are correct.
\end{answers}
\begin{solution}
\end{solution}

%%%%%%%%%%%%%%%%%%%%%%%%%%%%%%%%%%%%%%%%%
\question{1}
\variant
Consider the following program:
\begin{verbatim}
x=str(3*0.4)
\end{verbatim}
What is the \textbf{value} of x after this program is executed?
\begin{answers}
\answer \begin{verbatim}1.2\end{verbatim}
\answer \begin{verbatim}0\end{verbatim}
\correctanswer \begin{verbatim}"1.2"\end{verbatim}
\answer \begin{verbatim}"0"\end{verbatim}
\answer None of the other answers are correct.
\end{answers}
\begin{solution}
\end{solution}
\variant
Consider the following program:
\begin{verbatim}
x=str(1.2)*2
\end{verbatim}
What is the \textbf{value} of x after this program is executed?
\begin{answers}
\answer \begin{verbatim}2.4\end{verbatim}
\answer \begin{verbatim}"2.4"\end{verbatim}
\correctanswer \begin{verbatim}"1.21.2"\end{verbatim}
\answer \begin{verbatim}"1.2*2"\end{verbatim}
\answer None of the other answers are correct.
\end{answers}
\begin{solution}
\end{solution}
\variant
\begin{verbatim}
x=str(3)+"str(3)"
\end{verbatim}
What is the \textbf{value} of x after this program is executed?
\begin{answers}
\answer \begin{verbatim}"33"\end{verbatim}
\answer \begin{verbatim}33\end{verbatim}
\correctanswer \begin{verbatim}"3str(3)"\end{verbatim}
\answer \begin{verbatim}"333"\end{verbatim}
\answer None of the other answers are correct.
\end{answers}
\begin{solution}
\end{solution}


%%%%%%%%%%%%%%%%%%%%%%%%%%%%%%%%%%%%%%%%%
\question{1}
\variant
Consider the following program:
\begin{verbatim}
a=["A","C","C","I","O"]
a.sort()
a[0]=a[-1]
x=""
for e in a:
    x=x+e
\end{verbatim}
What is the \textbf{value} of x after this program is executed?
\begin{answers}
\correctanswer \begin{verbatim}"OCCIO"\end{verbatim}
\answer \begin{verbatim}"ACCOA"\end{verbatim}
\answer \begin{verbatim}"ACCIA"\end{verbatim}
\answer \begin{verbatim}"ICCOI"\end{verbatim}
\answer None of the other answers are correct.
\end{answers}
\begin{solution}
\end{solution}
\variant
Consider the following program:
\begin{verbatim}
a=["S","T","U","P","E","F","Y"]
a=a[0:4]
a.sort()
x=""
for e in a:
    x=e+x
\end{verbatim}
What is the \textbf{value} of x after this program is executed?
\begin{answers}
\answer \begin{verbatim}"PSTU"\end{verbatim}
\correctanswer \begin{verbatim}"UTSP"\end{verbatim}
\answer \begin{verbatim}"STUP"\end{verbatim}
\answer \begin{verbatim}"PUST"\end{verbatim}
\answer None of the other answers are correct.
\end{answers}
\begin{solution}
\end{solution}

%%%%%%%%%%%%%%%%%%%%%%%%%%%%%%%%%%%%%%%%%
\question{1}
\variant
Consider the following program:
\begin{verbatim}
a=3
b=4
if a==3:
    b=a
elif a==4:
    a=5
else:
    a=b
\end{verbatim}
What is the \textbf{value} of a after this program is executed?
\begin{answers}
\correctanswer \begin{verbatim}3\end{verbatim}
\answer \begin{verbatim}4\end{verbatim}
\answer \begin{verbatim}5\end{verbatim}
\answer \begin{verbatim}7\end{verbatim}
\answer None of the other answers are correct.
\end{answers}
\begin{solution}
\end{solution}
\variant
Consider the following program:
\begin{verbatim}
a=3
b=4
if a==3:
    a=b
elif a==4:
    a=5
else:
    b=a
\end{verbatim}
What is the \textbf{value} of a after this program is executed?
\begin{answers}
\answer \begin{verbatim}3\end{verbatim}
\correctanswer \begin{verbatim}4\end{verbatim}
\answer \begin{verbatim}5\end{verbatim}
\answer \begin{verbatim}7\end{verbatim}
\answer None of the other answers are correct.
\end{answers}
\begin{solution}
\end{solution}
\variant
Consider the following program:
\begin{verbatim}
a=3
b=4
if a!=b:
    a=b
elif a==4:
    a=5
else:
    b=a
\end{verbatim}
What is the \textbf{value} of a after this program is executed?
\begin{answers}
\answer \begin{verbatim}3\end{verbatim}
\correctanswer \begin{verbatim}4\end{verbatim}
\answer \begin{verbatim}5\end{verbatim}
\answer \begin{verbatim}7\end{verbatim}
\answer None of the other answers are correct.
\end{answers}
\begin{solution}
\end{solution}
%%%%%%%%%%%%%%%%%%%%%%%%%%%%%%%%%%%%%%%%%
\question{1}
\variant
Evaluate the following expression:
\begin{verbatim}
len("ABCD"[0:3])
\end{verbatim}
What value is produced?
\begin{answers}
\answer 1
\answer 2
\correctanswer 3
\answer 4
\end{answers}
\begin{solution}
\end{solution}
\variant
Evaluate the following expression:
\begin{verbatim}
len("ABCDE"[1:4])
\end{verbatim}
What value is produced?
\begin{answers}
\answer 1
\answer 5
\correctanswer 3
\answer 4
\end{answers}
\begin{solution}
\end{solution}

%%%%%%%%%%%%%%%%%%%%%%%%%%%%%%%%%%%%%%%%%
\question{1}
\variant
Evaluate the following expression:
\begin{verbatim}
(1/2)*2.0
\end{verbatim}
What value is produced?
\begin{answers}
\answer 1.0
\correctanswer 0.0
\answer 0.5
\answer 20.0
\end{answers}
\begin{solution}
\end{solution}
\variant
Evaluate the following expression:
\begin{verbatim}
(1*1.0)/2
\end{verbatim}
What value is produced?
\begin{answers}
\answer 1.0
\answer 0.0
\correctanswer 0.5
\answer 20.0
\end{answers}
\begin{solution}
\end{solution}
\variant
Evaluate the following expression:
\begin{verbatim}
1.0*(1/2)
\end{verbatim}
What value is produced?
\begin{answers}
\correctanswer 0.0
\answer 1.0
\answer 0.5
\answer 20.0
\end{answers}
\begin{solution}
\end{solution}
%%%%%%%%%%%%%%%%%%%%%%%%%%%%%%%%%%%%%%%%%
\question{1}
\variant
Evaluate the following expression:
\begin{verbatim}
[1,2]+[len("3")]
\end{verbatim}
What value is produced?
\begin{answers}
\correctanswer \begin{verbatim}[1,2,1]\end{verbatim}
\answer \begin{verbatim}[1,2,3]\end{verbatim}
\answer \begin{verbatim}[1,2,"3"]\end{verbatim}
\answer \begin{verbatim}[1,2,1,2,1,2]\end{verbatim}
\end{answers}
\begin{solution}
\end{solution}
\variant
Evaluate the following expression:
\begin{verbatim}
[1,2]*len("3")
\end{verbatim}
What value is produced?
\begin{answers}
\correctanswer \begin{verbatim}[1,2]\end{verbatim}
\answer \begin{verbatim}[1,2,3]\end{verbatim}
\answer \begin{verbatim}[1,2,1]\end{verbatim}
\answer \begin{verbatim}[1,2,1,2,1,2]\end{verbatim}
\end{answers}
\begin{solution}
\end{solution}

%%%%%%%%%%%%%%%%%%%%%%%%%%%%%%%%%%%%%%%%%
\question{1}
\variant
Examine the following program:
\begin{verbatim}
def fun(x):
    return 1
fun(3)
\end{verbatim}
What do we call x?
\begin{answers}
\correctanswer A parameter
\answer An argument
\answer A function
\answer A scope
\end{answers}
\begin{solution}
\end{solution}
\variant
Examine the following program:
\begin{verbatim}
def fun(x):
    return 1
fun(3)
\end{verbatim}
What do we call 3?
\begin{answers}
\answer A parameter
\correctanswer An argument
\answer A function
\answer A scope
\end{answers}
\begin{solution}
\end{solution}

%%%%%%%%%%%%%%%%%%%%%%%%%%%%%%%%%%%%%%%%%
\question{1}
\variant
Consider the following incomplete program.
\begin{verbatim}
sum=0
i=1
while ???<=1000:
    sum=sum+i*i
    i=i+1
print sum
\end{verbatim}
The program is intended to sum all of the perfect squares from 1 to 1000 (inclusive) and print the result. What should replace the three question marks to complete the program?
\begin{answers}
\correctanswer  \begin{verbatim}(i*i) \end{verbatim}
\answer  \begin{verbatim}(i**0.5) \end{verbatim}
\answer  \begin{verbatim}i % n \end{verbatim}
\answer  \begin{verbatim}i % sum \end{verbatim}
\end{answers}
\begin{solution}
\end{solution}
\variant
Consider the following incomplete program.
\begin{verbatim}
prime=True
for i in range(2,n):
    if ???:
       prime=False
if prime:
    print "n is prime."
\end{verbatim}
The program is intended to print a message if an integer n is prime. What should replace the three question marks to complete the program?
\begin{answers}
\correctanswer  \begin{verbatim}(n % i)==0 \end{verbatim}
\answer  \begin{verbatim}i*i < n \end{verbatim}
\answer  \begin{verbatim}i**.5 < n \end{verbatim}
\answer  \begin{verbatim}(i / n) == 0 \end{verbatim}
\end{answers}
\begin{solution}
\end{solution}

%%%%%%%%%%%%%%%%%%%%%%%%%%%%%%%%%%%%%%%%%
\question{1}
\variant
Consider the following incomplete function.
\begin{verbatim}
def only_vowel(s):
    for c in s:
        if ???:
            return False
    return True
\end{verbatim}
The function is intended to return True if and only if the input string s consists entirely of characters representing lower case vowels. For example, \texttt{only\_vowels("iaou")} should return True, but \texttt{only\_vowels("eAo2i")} should return False, because A is not lower case and 2 is not a vowel. What should replace the three question marks to complete the function?
\begin{answers}
\answer  \begin{verbatim} c in "aeiou" \end{verbatim}
\answer  \begin{verbatim} c.lower() in "aeiou" \end{verbatim}
\answer  \begin{verbatim} c.lower() not in "aeiou" \end{verbatim}
\correctanswer  \begin{verbatim}  c not in "aeiou" \end{verbatim}
\end{answers}
\begin{solution}
\end{solution}
\question{1}
\variant
Consider the following incomplete function.
\begin{verbatim}
def no_evens(s):
    for c in s:
        if ???:
            return False
    return True
\end{verbatim}
The function is intended to return True if and only if the input string s contains \textbf{no} characters representing even digits. For example, \texttt{no\_even("AB13")} should return True, but \texttt{no\_even("132")} should return False, because 2 is not an even digit. What should replace the three question marks to complete the function?
\begin{answers}
\correctanswer  \begin{verbatim} (int(c)%2)!=0 and c in "0123456789" \end{verbatim}
\answer  \begin{verbatim} c not in "02468" \end{verbatim}
\answer  \begin{verbatim} (int(c)%2)!=0 or c in "0123456789" \end{verbatim}
\answer  \begin{verbatim}  c in "13579" \end{verbatim}
\end{answers}
\begin{solution}
\end{solution}

%%%%%%%%%%%%%%%%%%%%%%%%%%%%%%%%%%%%%%%%%
\question{1}
\variant
Consider the following incomplete program.
\begin{verbatim}
sum=0
???:
    sum=sum+i

\end{verbatim}
The program is intended to sum all of the integers between 1 and 100 (inclusive). What should replace the three question marks to complete the program?
\begin{answers}
\answer  \begin{verbatim}for i in range(0,100)\end{verbatim}
\answer  \begin{verbatim}while i<=100 \end{verbatim}
\correctanswer  \begin{verbatim}for i in range(1,101) \end{verbatim}
\answer  \begin{verbatim}while i*i <=100 \end{verbatim}
\end{answers}
\begin{solution}
\end{solution}
\variant
Consider the following incomplete program.
\begin{verbatim}
sum=0
for i in range(0,100):
    ???

\end{verbatim}
The program is intended to sum all of the integers between 1 and 100 (inclusive). What should replace the three question marks to complete the program?
\begin{answers}
\answer  \begin{verbatim}sum=sum+1\end{verbatim}
\answer  \begin{verbatim}sum+1=sum \end{verbatim}
\answer  \begin{verbatim}sum=sum+i \end{verbatim}
\correctanswer  \begin{verbatim}sum=sum+i+1 \end{verbatim}
\end{answers}
\begin{solution}
\end{solution}
%%%%%%%%%%%%%%%%%%%%%%%%%%%%%%%%%%%%%%%%%
\question{1}
\variant
Assume the variable x is a string. Which of the following expressions will retrieve the third character in the string?
\begin{answers}
\answer  \begin{verbatim}[3]x\end{verbatim}
\answer  \begin{verbatim}x*3 \end{verbatim}
\correctanswer  \begin{verbatim}x[2] \end{verbatim}
\answer  \begin{verbatim}x[3]\end{verbatim}
\answer  \begin{verbatim}x+3\end{verbatim}
\end{answers}
\begin{solution}
\end{solution}
\variant
Assume the variable x is a string. Which of the following expressions will retrieve the last character in the string?
\begin{answers}
\correctanswer  \begin{verbatim}x[-1]\end{verbatim}
\answer  \begin{verbatim}[0]x \end{verbatim}
\answer  \begin{verbatim}x[:-1] \end{verbatim}
\answer  \begin{verbatim}x[1] \end{verbatim}
\answer  \begin{verbatim}[1]x\end{verbatim}
\end{answers}
\begin{solution}
\end{solution}
\variant
Assume the variable x is a string. Which of the following expressions will retrieve the first character in the string?
\begin{answers}
\answer  \begin{verbatim}[1]x\end{verbatim}
\answer  \begin{verbatim}[-1]x \end{verbatim}
\correctanswer  \begin{verbatim}x[0*-1] \end{verbatim}
\answer  \begin{verbatim}x[-1]\end{verbatim}
\end{answers}
\begin{solution}
\end{solution}

%%%%%%%%%%%%%%%%%%%%%%%%%%%%%%%%%%%%%%%%%
\question{1}
\variant
If we evaluate the following expressions, which one will produce a value of type integer?
\begin{answers}
\answer  \begin{verbatim}"ABCD".lower()\end{verbatim}
\answer  \begin{verbatim}str(333)[3] \end{verbatim}
\correctanswer  \begin{verbatim}len("ABCD")\end{verbatim}
\answer  \begin{verbatim}"3"*3\end{verbatim}
\end{answers}
\begin{solution}
\end{solution}
\variant
If we evaluate the following expressions, which one will produce a value of type string?
\begin{answers}
\answer  \begin{verbatim}len("33"+"33")\end{verbatim}
\correctanswer  \begin{verbatim}["1","2","3"][1] \end{verbatim}
\answer  \begin{verbatim}["A","B","C"].sort()\end{verbatim}
\answer  \begin{verbatim}(1+3+5)%4 \end{verbatim}
\end{answers}
\begin{solution}
\end{solution}

%%%%%%%%%%%%%%%%%%%%%%%%%%%%%%%%%%%%%%%%%
\question{1}
\variant
Consider the following program:
\begin{verbatim}
x=["tick","tock"]
x[0]="tick" in x
if x[0]:
    x=1
\end{verbatim}
What is the \textbf{type} of x after the program is run?
\begin{answers}
\answer None
\correctanswer Integer
\answer List
\answer String
\answer None of the other answers are correct.
\end{answers}
\begin{solution}
\end{solution}
\variant
Consider the following program:
\begin{verbatim}
x=["tick","tock"]
x[0]=len(x[1])
if x[0]<4:
    x="A"
\end{verbatim}
What is the \textbf{type} of x after the program is run?
\begin{answers}
\answer None
\answer Integer
\correctanswer List
\answer String
\answer None of the other answers are correct.
\end{answers}
\begin{solution}
\end{solution}


\end{document}
