%!TEX program = xelatex
\documentclass[11pt]{beamer}

\usepackage{amsfonts}
\usepackage{amsmath}
\usepackage{blindtext}
\usepackage{enumitem}

\usetheme{SaoPaulo}

\title{Python Basics!}
\subtitle{functions, scope}
\author{CS101 Lecture \#4}
\date{2016-08-31}

\setcounter{showSlideNumbers}{1}

\begin{document}
  \setcounter{showProgressBar}{0}
  \setcounter{showSlideNumbers}{0}

%%%%%%%%%%%%%%%%%%%%%%%%%%%%%%%%%%%%%%%%%%%%%%%%%%%%%%%%%%%%%%%%%%%%%%%%%%%%%%%%
\frame{\titlepage}

%%%%%%%%%%%%%%%%%%%%%%%%%%%%%%%%%%%%%%%%%%%%%%%%%%%%%%%%%%%%%%%%%%%%%%%%%%%%%%%%
\setcounter{framenumber}{0}
\setcounter{showProgressBar}{1}
\setcounter{showSlideNumbers}{1}

%%%%%%%%%%%%%%%%%%%%%%%%%%%%%%%%%%%%%%%%%%%%%%%%%%%%%%%%%%%%%%%%%%%%%%%%%%%%%%%%
\section{Administrivia}

%%%%%%%%%%%%%%%%%%%%%%%%%%%%%%%%%%%%%%%%%%%%%%%%%%%%%%%%%%%%%%%%%%%%%%%%%%%%%%%%
\begin{frame}
  \frametitle{Administrivia}
  \Enlarge
  \begin{itemize}
  \myitem  Register your i>clickers on the course Compass page---attendance counts from today! \pause
    \begin{itemize}
    \mysubitem  Don't panic! \pause
    \end{itemize}
  \myitem  Complete Homework \#1 before 6:00 p.m. today. \pause
  \myitem  Homework \#2 is due Friday Sep.\ 9. \pause
  \myitem  No lab next week (Labor Day).
  \end{itemize}
\end{frame}

%%%%%%%%%%%%%%%%%%%%%%%%%%%%%%%%%%%%%%%%%%%%%%%%%%%%%%%%%%%%%%%%%%%%%%%%%%%%%%%%
\section{Warmup Quiz}

%%%%%%%%%%%%%%%%%%%%%%%%%%%%%%%%%%%%%%%%%%%%%%%%%%%%%%%%%%%%%%%%%%%%%%%%%%%%%%%%
\begin{frame}[fragile]
  \frametitle{Question \#1}
  \Enlarge

  \begin{semiverbatim}
x = "3"
y = 10 % 4
print(x * y)
  \end{semiverbatim}
  What does this program print?
  \begin{enumerate}[label=\Alph*]
  \item  6
  \item  2
  \item  33
  \item  32
  \end{enumerate}
\end{frame}

%%%%%%%%%%%%%%%%%%%%%%%%%%%%%%%%%%%%%%%%%%%%%%%%%%%%%%%%%%%%%%%%%%%%%%%%%%%%%%%%
\begin{frame}[fragile]
  \frametitle{Question \#2}
  \Enlarge

  \begin{semiverbatim}
c = (10 + 5j)
i = 25
r = c.real + i
  \end{semiverbatim}
  What is the type and value of \texttt{r}?
  \begin{enumerate}[label=\Alph*]
  \item  \texttt{int}, \texttt{35}
  \item  \texttt{complex}, \texttt{35 + 5j}
  \item  \texttt{float}, \texttt{35.0}
  \item  \texttt{complex}, \texttt{35 + 0j}
  \end{enumerate}
\end{frame}

%%%%%%%%%%%%%%%%%%%%%%%%%%%%%%%%%%%%%%%%%%%%%%%%%%%%%%%%%%%%%%%%%%%%%%%%%%%%%%%%
\begin{frame}
  \frametitle{Question \#3}
  \Enlarge

  Which of these expressions is most likely to cause an \textbf{overflow}?
  \begin{enumerate}[label=\Alph*]
  \item  \texttt{10 ** 100000}
  \item  \texttt{"10" * 100000}
  \item  \texttt{10.0 ** 100000}
  \item  \texttt{"10" ** 100000}
  \item  None of the above
  \end{enumerate}
\end{frame}

%%%%%%%%%%%%%%%%%%%%%%%%%%%%%%%%%%%%%%%%%%%%%%%%%%%%%%%%%%%%%%%%%%%%%%%%%%%%%%%%
\begin{frame}[fragile]
  \frametitle{Question \#4}
  \Enlarge

  \begin{semiverbatim}
x = "10"
y = "%i"
print( (x+y) % 2)
  \end{semiverbatim}
  What does this program print?
  \begin{enumerate}[label=\Alph*]
  \item  \texttt{102}
  \item  \texttt{1111}
  \item  \texttt{1010}
  \item  None of the above
  \end{enumerate}
\end{frame}

%%%%%%%%%%%%%%%%%%%%%%%%%%%%%%%%%%%%%%%%%%%%%%%%%%%%%%%%%%%%%%%%%%%%%%%%%%%%%%%%
\section{Data Types---A Few Points}

%%%%%%%%%%%%%%%%%%%%%%%%%%%%%%%%%%%%%%%%%%%%%%%%%%%%%%%%%%%%%%%%%%%%%%%%%%%%%%%%
\begin{frame}
  \frametitle{Complex numbers, $\mathbb{C}$}
  \Enlarge

  \begin{itemize}
  \myitem  Represent numbers with an imaginary component.
  \myitem  Use \texttt{j} for $i$: \\
    \textcolor{CS101GradBot}{\texttt{z = 1.0 + 1j}} \pause
  \myitem  \texttt{z.real + z.imag * 1j}
  \end{itemize}
\end{frame}

%%%%%%%%%%%%%%%%%%%%%%%%%%%%%%%%%%%%%%%%%%%%%%%%%%%%%%%%%%%%%%%%%%%%%%%%%%%%%%%%
\begin{frame}
  \frametitle{Strings}
  \Enlarge

  \begin{itemize}
  \myitem  As a literal:  text surrounded by quotes.
    \begin{itemize}
    \mysubitem  \texttt{"DEEP"}
    \end{itemize}
  \myitem  Each symbol is a character.
  \myitem  Unlike numeric types, strings vary in length.
  \end{itemize}
\end{frame}

%%%%%%%%%%%%%%%%%%%%%%%%%%%%%%%%%%%%%%%%%%%%%%%%%%%%%%%%%%%%%%%%%%%%%%%%%%%%%%%%
\begin{frame}
  \frametitle{String operations}
  \Enlarge

  \begin{itemize}
  \myitem  \textbf{Concatenation}:  combine two strings
    \begin{itemize}
    \mysubitem  Uses the \texttt{+} symbol
    \mysubitem  \texttt{'RACE' + 'CAR'}
    \end{itemize}
  \myitem  \textbf{Repetition}:  repeat a string
    \begin{itemize}
    \mysubitem  Uses the \texttt{*}
    \mysubitem  \texttt{'HELLO '*10}
    \end{itemize}
  \myitem  \textbf{Formatting}:  used to encode other data as string
    \begin{itemize}
    \mysubitem  Uses \texttt{\%} symbol
    \end{itemize}
  \end{itemize}
\end{frame}

%%%%%%%%%%%%%%%%%%%%%%%%%%%%%%%%%%%%%%%%%%%%%%%%%%%%%%%%%%%%%%%%%%%%%%%%%%%%%%%%
\begin{frame}
  \frametitle{Formatting operator}
  \Enlarge

  \begin{itemize}
  \myitem  Creates string with value inserted
    \begin{itemize}
    \mysubitem  Formats nicely
    \mysubitem  Requires indicator of type inside of string
      \begin{tabular}{*{27}{ll}}
        \texttt{"\%i"} & \texttt{int} \\
        \texttt{"\%f"} & \texttt{float} \\
        \texttt{"\%e"} & \texttt{float} (scientific notation) \\
        \texttt{"\%s"} & \texttt{str}
      \end{tabular}
    \end{itemize}
  \end{itemize}
\end{frame}

%%%%%%%%%%%%%%%%%%%%%%%%%%%%%%%%%%%%%%%%%%%%%%%%%%%%%%%%%%%%%%%%%%%%%%%%%%%%%%%%
\begin{frame}[fragile]
  \frametitle{Example}
  \Enlarge

  \begin{semiverbatim}
print( "An integer:  \%i" \% 7 )
print( "A float:     \%f" \% 7.0 )
print( "A float:     \%e" \% 7.0 )
print( "A string:    \%s" \% 'seven' )
  \end{semiverbatim}
\end{frame}

%%%%%%%%%%%%%%%%%%%%%%%%%%%%%%%%%%%%%%%%%%%%%%%%%%%%%%%%%%%%%%%%%%%%%%%%%%%%%%%%
\begin{frame}[fragile]
  \frametitle{Indexing operator \textbf{[]}}
  \Enlarge

  \begin{itemize}
  \myitem  Extracts single character
\begin{semiverbatim}
a = "FIRE"
a[0]
\end{semiverbatim}
  \myitem  The integer is the index. \pause
  \myitem  \textcolor{red}{We count from zero!} \pause
  \myitem  If negative, counts down from end. \pause
  \myitem  Does this work on other data types like \texttt{int}?
  \end{itemize}
\end{frame}

%%%%%%%%%%%%%%%%%%%%%%%%%%%%%%%%%%%%%%%%%%%%%%%%%%%%%%%%%%%%%%%%%%%%%%%%%%%%%%%%
\begin{frame}[fragile]
  \frametitle{Slicing operator \textbf{:}}
  \Enlarge

  \begin{itemize}
  \myitem  Extracts range of characters (\emph{substring}) \pause
  \myitem  Range specified inside of indexing operator \pause
\begin{semiverbatim}
a = "FIREHOUSE"
a[0:4]
\end{semiverbatim} \pause
  \myitem  Can be a bit tricky at first:
    \begin{itemize}
    \mysubitem  Includes character at first index
    \mysubitem  Excludes character at last index
    \end{itemize}
  \end{itemize}
\end{frame}

%%%%%%%%%%%%%%%%%%%%%%%%%%%%%%%%%%%%%%%%%%%%%%%%%%%%%%%%%%%%%%%%%%%%%%%%%%%%%%%%
\begin{frame}[fragile]
  \frametitle{Example}
  \Enlarge

  \begin{semiverbatim}
alpha = "ABCDE"
x = alpha[1:3]
  \end{semiverbatim}
  What is the value of \texttt{x}?
  \begin{enumerate}[label=\Alph*]
  \item  \texttt{'AB'}
  \item  \texttt{'ABC'}
  \item  \texttt{'BC'}
  \item  \texttt{'BCD'}
  \item  \texttt{'CD'}
  \end{enumerate}
\end{frame}

%%%%%%%%%%%%%%%%%%%%%%%%%%%%%%%%%%%%%%%%%%%%%%%%%%%%%%%%%%%%%%%%%%%%%%%%%%%%%%%%
\section{Functions}

%%%%%%%%%%%%%%%%%%%%%%%%%%%%%%%%%%%%%%%%%%%%%%%%%%%%%%%%%%%%%%%%%%%%%%%%%%%%%%%%
\begin{frame}
  \frametitle{And now for something different...}
  \Enlarge

  \begin{itemize}
  \myitem  A \emph{function} is a small program (block of code) we can run within Python. \pause
    \begin{itemize}
    \mysubitem  Saves us from rewriting code
    \mysubitem  Don't reinvent the wheel!
    \end{itemize} \pause
  \myitem  Analogy:  Functions are more verbs. \pause
  \myitem  Also called subroutine or procedure.
  \end{itemize}
\end{frame}

%%%%%%%%%%%%%%%%%%%%%%%%%%%%%%%%%%%%%%%%%%%%%%%%%%%%%%%%%%%%%%%%%%%%%%%%%%%%%%%%
\begin{frame}
  \frametitle{Function calls}
  \Enlarge

  \begin{itemize}
  \myitem  When we want to execute a function, we call or invoke it. \pause
  \myitem  Use name of the function with parentheses. \pause
    \begin{itemize}
    \mysubitem  \texttt{print()}
    \end{itemize} \pause
  \myitem  Many functions come built-in to Python or in the standard library. \pause
  \myitem  Others we will compose at need.
  \end{itemize}
\end{frame}

%%%%%%%%%%%%%%%%%%%%%%%%%%%%%%%%%%%%%%%%%%%%%%%%%%%%%%%%%%%%%%%%%%%%%%%%%%%%%%%%
\begin{frame}
  \frametitle{Arguments}
  \Enlarge

  \begin{itemize}
  \myitem  Functions can act on data. \pause
  \myitem  \emph{Arguments} are the input to functions. \pause
  \myitem  Functions can return a value.  (fruitful) \pause
  \myitem  Return values are the output of a function. \pause
    \begin{itemize}
    \mysubitem  \texttt{print('10')} \pause
    \mysubitem  \texttt{len('Rex Kwon Do')} \pause
    \mysubitem  \texttt{abs(-123)} \pause
    \end{itemize}
  \end{itemize}
\end{frame}

%%%%%%%%%%%%%%%%%%%%%%%%%%%%%%%%%%%%%%%%%%%%%%%%%%%%%%%%%%%%%%%%%%%%%%%%%%%%%%%%
\begin{frame}
  \frametitle{Arguments}
  \Enlarge

  \begin{itemize}
  \myitem  \emph{Arguments} are values passed \emph{to} a function. \pause
  \myitem  A function can accept zero to many arguments. \pause
  \myitem  Multiple arguments are separated by commas:
    \begin{itemize}
    \mysubitem  \texttt{min( 1,4,5 )}
    \mysubitem  \texttt{max( 1,4,5 )}
    \end{itemize}
  \end{itemize}
\end{frame}

%%%%%%%%%%%%%%%%%%%%%%%%%%%%%%%%%%%%%%%%%%%%%%%%%%%%%%%%%%%%%%%%%%%%%%%%%%%%%%%%
\begin{frame}
  \frametitle{Type conversion.}
  \Enlarge

  \begin{itemize}
  \myitem  A set of built-in functions to convert data from one type to another. \pause
    \begin{itemize}
    \mysubitem  \texttt{float( "0.3" )}
    \mysubitem  \texttt{str( 3 + 5j )}
    \end{itemize} \pause
  \myitem  Be careful of nonsense:
    \begin{itemize}
    \mysubitem  \texttt{int( "Rex" )}
    \mysubitem  \texttt{int( 3 + 5j )}
    \end{itemize}
  \myitem  Also called subroutine or procedure.
  \end{itemize}
\end{frame}

%%%%%%%%%%%%%%%%%%%%%%%%%%%%%%%%%%%%%%%%%%%%%%%%%%%%%%%%%%%%%%%%%%%%%%%%%%%%%%%%
\begin{frame}
  \frametitle{User input}
  \Enlarge

  \begin{itemize}
  \myitem  \texttt{input} is a built-in function. \pause
  \myitem  Argument:  string prompting user \pause
  \myitem  Return value:  input from user (as \texttt{str})
  \end{itemize}
\end{frame}

%%%%%%%%%%%%%%%%%%%%%%%%%%%%%%%%%%%%%%%%%%%%%%%%%%%%%%%%%%%%%%%%%%%%%%%%%%%%%%%%
\begin{frame}
  \frametitle{Goal}
  \Enlarge

  \begin{itemize}
  \myitem  A program should achieve a goal. \pause
  \myitem  Next time we will write our first nontrivial program.
  \end{itemize}
\end{frame}

%%%%%%%%%%%%%%%%%%%%%%%%%%%%%%%%%%%%%%%%%%%%%%%%%%%%%%%%%%%%%%%%%%%%%%%%%%%%%%%%
\section{Reminders}

%%%%%%%%%%%%%%%%%%%%%%%%%%%%%%%%%%%%%%%%%%%%%%%%%%%%%%%%%%%%%%%%%%%%%%%%%%%%%%%%
\begin{frame}
  \frametitle{Reminders}
  \Enlarge

  \begin{itemize}
  \myitem  Homework \#1 due today, Aug.\ 31, 5:00 p.m.
  \myitem  Homework \#2 due Friday, Sep.\ 9, 5:00 p.m.
  \myitem  No class Monday, Sep.\ 5 (Labor Day).
  \myitem  No lab next week!
  \end{itemize}
\end{frame}

\end{document}
