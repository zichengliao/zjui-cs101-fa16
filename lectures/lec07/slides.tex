%!TEX program = xelatex
\documentclass[11pt]{beamer}

\usepackage{amsfonts}
\usepackage{amsmath}
\usepackage{blindtext}
\usepackage{enumitem}

\usetheme{SaoPaulo}

\title{Python Basics!}
\subtitle{lists and loops}
\author{CS101 Lecture \#7}
\date{2016-10-19}

\setcounter{showSlideNumbers}{1}

\begin{document}
  \setcounter{showProgressBar}{0}
  \setcounter{showSlideNumbers}{0}

%%%%%%%%%%%%%%%%%%%%%%%%%%%%%%%%%%%%%%%%%%%%%%%%%%%%%%%%%%%%%%%%%%%%%%%%%%%%%%%%
\frame{\titlepage}

%%%%%%%%%%%%%%%%%%%%%%%%%%%%%%%%%%%%%%%%%%%%%%%%%%%%%%%%%%%%%%%%%%%%%%%%%%%%%%%%
\setcounter{framenumber}{0}
\setcounter{showProgressBar}{1}
\setcounter{showSlideNumbers}{1}

%%%%%%%%%%%%%%%%%%%%%%%%%%%%%%%%%%%%%%%%%%%%%%%%%%%%%%%%%%%%%%%%%%%%%%%%%%%%%%%%
\section{Administrivia}

%%%%%%%%%%%%%%%%%%%%%%%%%%%%%%%%%%%%%%%%%%%%%%%%%%%%%%%%%%%%%%%%%%%%%%%%%%%%%%%%
\begin{frame}
  \frametitle{Administrivia}
  \Enlarge
  \begin{itemize}
  \myitem  Homework \#2 is due Wed Oct.\ 19 today.
  \myitem  Homework \#3 is due Wed Oct.\ 26.
  \myitem  Midterm \#1 will be on the day of the 12th lecture, covering through Lecture \#11.  
  \end{itemize}
\end{frame}


%%%%%%%%%%%%%%%%%%%%%%%%%%%%%%%%%%%%%%%%%%%%%%%%%%%%%%%%%%%%%%%%%%%%%%%%%%%%%%%%
\section{Container Data Types}

%%%%%%%%%%%%%%%%%%%%%%%%%%%%%%%%%%%%%%%%%%%%%%%%%%%%%%%%%%%%%%%%%%%%%%%%%%%%%%%%
\begin{frame}[fragile]
  \frametitle{Example}
  \Enlarge

  \begin{semiverbatim}
colors = [ 'red', 'yellow', 'blue',
           'jale', 'ulfire' ]
for color in colors:
    print( color.title() )
  \end{semiverbatim}
\end{frame}

%%%%%%%%%%%%%%%%%%%%%%%%%%%%%%%%%%%%%%%%%%%%%%%%%%%%%%%%%%%%%%%%%%%%%%%%%%%%%%%%
\begin{frame}[fragile]
  \frametitle{\texttt{list} data type}
  \Enlarge

  \begin{itemize}
  \myitem  The \texttt{list} type represents an ordered collection of items. %%\pause
  \myitem  \texttt{list} is an \emph{iterable} and a \emph{container}. %%\pause
  \myitem  Containers hold values of any type (doesn't have to be the same).
  \end{itemize}
\end{frame}

%%%%%%%%%%%%%%%%%%%%%%%%%%%%%%%%%%%%%%%%%%%%%%%%%%%%%%%%%%%%%%%%%%%%%%%%%%%%%%%%
\begin{frame}[fragile]
  \frametitle{\texttt{list} statement}
  \Enlarge

  \begin{itemize}
  \myitem  We create a \texttt{list} as follows:
    \begin{itemize}
    \mysubitem  opening bracket \texttt{[}
    \mysubitem  one or more comma-separated data values
    \mysubitem  closing bracket \texttt{]}
    \end{itemize}
  \end{itemize}
\end{frame}

%%%%%%%%%%%%%%%%%%%%%%%%%%%%%%%%%%%%%%%%%%%%%%%%%%%%%%%%%%%%%%%%%%%%%%%%%%%%%%%%
\begin{frame}[fragile]
  \frametitle{\texttt{list} statement}
  \Enlarge

  \begin{itemize}
  \myitem  \texttt{list}s work a bit like strings:
    \begin{semiverbatim}
x = [ 10, 3.14, "Ride" ]

print( x[1] )
print( x[1:3] )
print( len(x) )
    \end{semiverbatim}
  \end{itemize}
\end{frame}

%%%%%%%%%%%%%%%%%%%%%%%%%%%%%%%%%%%%%%%%%%%%%%%%%%%%%%%%%%%%%%%%%%%%%%%%%%%%%%%%
\begin{frame}[fragile]
  \frametitle{\texttt{list} statement}
  \Enlarge

  \begin{itemize}
  \myitem  But strings are \emph{immutable} (we cannot change contents without creating a new string): %%\pause
  \end{itemize}
  \begin{semiverbatim}
s = "good advise"
s[9] = 'c'                 # nope

s = s[:9] + 'c' + s[9:]    # this way
  \end{semiverbatim}
\end{frame}

%%%%%%%%%%%%%%%%%%%%%%%%%%%%%%%%%%%%%%%%%%%%%%%%%%%%%%%%%%%%%%%%%%%%%%%%%%%%%%%%
\begin{frame}[fragile]
  \frametitle{\texttt{list} statement}
  \Enlarge

  \begin{itemize}
  \myitem  We \emph{can} change \texttt{list} content---they are \emph{mutable}. %%\pause
  \end{itemize}
  \begin{columns}
  \begin{column}{0.6\textwidth}
    \begin{semiverbatim}
    x = [ 4,1,2,3 ]
    x[3] = -2
    x.append(5)
    del x[1]
    x.sort()
    \end{semiverbatim}
  \end{column}
  \begin{column}{0.4\textwidth}
  ~ \\
  \textcolor{red}{$\leftarrow$ item assignment}
  ~ \\ ~ \\ ~ \\ ~ \\
  \end{column}
  \end{columns}
\end{frame}

%%%%%%%%%%%%%%%%%%%%%%%%%%%%%%%%%%%%%%%%%%%%%%%%%%%%%%%%%%%%%%%%%%%%%%%%%%%%%%%%
\section{Loops}

%%%%%%%%%%%%%%%%%%%%%%%%%%%%%%%%%%%%%%%%%%%%%%%%%%%%%%%%%%%%%%%%%%%%%%%%%%%%%%%%
\begin{frame}[fragile]
  \frametitle{Loops}
  \Enlarge

  \begin{itemize}
  \myitem  We frequently need to process each value in a set of values. %%\pause
  \myitem  Two kinds:  \texttt{while} and \texttt{for}
  \end{itemize}
\end{frame}

%%%%%%%%%%%%%%%%%%%%%%%%%%%%%%%%%%%%%%%%%%%%%%%%%%%%%%%%%%%%%%%%%%%%%%%%%%%%%%%%
\begin{frame}[fragile]
  \frametitle{Example:  \texttt{while} Loop}
  \Enlarge

  \begin{semiverbatim}
number = 10
while number > 0:
    print(number)
    number = number - 1
print('Blast off!')
  \end{semiverbatim}
\end{frame}

%%%%%%%%%%%%%%%%%%%%%%%%%%%%%%%%%%%%%%%%%%%%%%%%%%%%%%%%%%%%%%%%%%%%%%%%%%%%%%%%
\begin{frame}[fragile]
  \frametitle{Defining loops:  \texttt{while}}
  \Enlarge

  \begin{itemize}
  \myitem  A \texttt{while} loop has only:
    \begin{itemize}
    \mysubitem  the keyword \texttt{while}
    \mysubitem  a logical comparison (\texttt{bool}-valued result)
    \mysubitem  a \textbf{block} of code
    \end{itemize}
  \end{itemize}
\end{frame}

%%%%%%%%%%%%%%%%%%%%%%%%%%%%%%%%%%%%%%%%%%%%%%%%%%%%%%%%%%%%%%%%%%%%%%%%%%%%%%%%
\begin{frame}[fragile]
  \frametitle{Example}
  \Enlarge

%  The following code should increment \texttt{x} if the hundreds place contains a zero:
  \begin{semiverbatim}
x = 3
while x > 0:
    print("Hello")
    x -= 1
  \end{semiverbatim}
  How many times is \texttt{'Hello'} printed?
  \begin{enumerate}[label=\Alph*]
  \item  zero
  \item  once
  \item  twice
  \item  thrice
  \item  four times
  \end{enumerate}
\end{frame}

%%%%%%%%%%%%%%%%%%%%%%%%%%%%%%%%%%%%%%%%%%%%%%%%%%%%%%%%%%%%%%%%%%%%%%%%%%%%%%%%
\begin{frame}[fragile]
  \frametitle{String comparison methods}
  \Enlarge

  \begin{itemize}
  \myitem  These produce Boolean output.
    \begin{tabular}{ll}
    \texttt{isdigit()} & Does a string contain \\
    & only numbers (digits); >= 1 character?\\
    \texttt{isalpha()} & Does a string contain \\
    & only text (alphabetic); >= 1 character? \\
    \texttt{islower()} & Does a string contain \\
    & only lower-case letters; >= 1 character? \\
    \texttt{isupper()} & Does a string contain \\
    & only upper-case letters; >= 1 character?
    \end{tabular}
  \end{itemize}
\end{frame}

%%%%%%%%%%%%%%%%%%%%%%%%%%%%%%%%%%%%%%%%%%%%%%%%%%%%%%%%%%%%%%%%%%%%%%%%%%%%%%%%
\begin{frame}[fragile]
  \frametitle{Example:  String comparison methods}
  \Enlarge

  \begin{semiverbatim}
answer = input( 'How do you feel?  ' )
if not answer.isalpha():
    print( "I don't understand." )
else:
    print( "Ah, you feel %s." % answer )
  \end{semiverbatim}
\end{frame}

%%%%%%%%%%%%%%%%%%%%%%%%%%%%%%%%%%%%%%%%%%%%%%%%%%%%%%%%%%%%%%%%%%%%%%%%%%%%%%%%
\begin{frame}[fragile]
  \frametitle{Exercise}
  \Enlarge

  Write a program for a user to create a new password.  The program should accept a password attempt from the user and check it with the function \texttt{validate\_password}.  If the password is valid, the program ends.  If the password is invalid, the program asks for a new attempt, repeating until the user enters a valid password.
\end{frame}

%%%%%%%%%%%%%%%%%%%%%%%%%%%%%%%%%%%%%%%%%%%%%%%%%%%%%%%%%%%%%%%%%%%%%%%%%%%%%%%%
\begin{frame}[fragile]
  \frametitle{Solution}
  \Enlarge

  \begin{semiverbatim}
pwd = input("Enter a password:  ")
while not validate_password(pwd):
    pwd = input("INVALID!  Try again:  ")
print("Your password is valid.")
  \end{semiverbatim}
\end{frame}

%%%%%%%%%%%%%%%%%%%%%%%%%%%%%%%%%%%%%%%%%%%%%%%%%%%%%%%%%%%%%%%%%%%%%%%%%%%%%%%%
\begin{frame}[fragile]
  \frametitle{Infinite loops}
  \Enlarge

  \begin{itemize}
  \myitem  Make sure that your code always has a way to end! %%\pause
  \begin{semiverbatim}
while True:
    print('Hello!')
  \end{semiverbatim}
  \end{itemize}
\end{frame}

%%%%%%%%%%%%%%%%%%%%%%%%%%%%%%%%%%%%%%%%%%%%%%%%%%%%%%%%%%%%%%%%%%%%%%%%%%%%%%%%
\begin{frame}[fragile]
  \frametitle{Infinite loops}
  \Enlarge

  \begin{itemize}
  \myitem  Make sure that your code always has a way to end!
  \begin{semiverbatim}
while True:
    print('Hello!')
  \end{semiverbatim}
  \myitem  \textcolor{red}{Use \texttt{Ctrl}+\texttt{C} to break free.}
  \end{itemize}
\end{frame}

%%%%%%%%%%%%%%%%%%%%%%%%%%%%%%%%%%%%%%%%%%%%%%%%%%%%%%%%%%%%%%%%%%%%%%%%%%%%%%%%
\begin{frame}[fragile]
  \frametitle{Accumulator pattern}
  \Enlarge

  \begin{itemize}
  \myitem  \emph{Design patterns} are common structures we encounter in writing code. %%\pause
  \myitem  The \emph{accumulator} pattern uses an accumulator variable to track a result inside of a loop: %%\pause
  \begin{semiverbatim}
i = 0
sum = 0
while i <= 4:
    sum += i
    i += 1
  \end{semiverbatim}
  \end{itemize}
\end{frame}

%%%%%%%%%%%%%%%%%%%%%%%%%%%%%%%%%%%%%%%%%%%%%%%%%%%%%%%%%%%%%%%%%%%%%%%%%%%%%%%%
\begin{frame}[fragile]
  \frametitle{Example}
  \Enlarge

  \begin{semiverbatim}
i = 0
sum = 0
while i <= 4:
    sum += i
    i += 1
  \end{semiverbatim}
  What is the value of \texttt{sum}?
  \begin{enumerate}[label=\Alph*]
  \item  \texttt{6}
  \item  \texttt{10}
  \item  \texttt{15}
  \item  None of the above.
  \end{enumerate}
\end{frame}

%%%%%%%%%%%%%%%%%%%%%%%%%%%%%%%%%%%%%%%%%%%%%%%%%%%%%%%%%%%%%%%%%%%%%%%%%%%%%%%%
\begin{frame}[fragile]
  \frametitle{Example}
  \Enlarge

  \begin{semiverbatim}
i = 0
sum = 0
while i < 7:
    if (i % 2) == 1:
        sum += i
    i += 1
  \end{semiverbatim}
  What is the value of \texttt{sum}?
  \begin{enumerate}[label=\Alph*]
  \item  \texttt{9} % answer
  \item  \texttt{12}
  \item  \texttt{16}
  \item  \texttt{21}
  \end{enumerate}
\end{frame}

%%%%%%%%%%%%%%%%%%%%%%%%%%%%%%%%%%%%%%%%%%%%%%%%%%%%%%%%%%%%%%%%%%%%%%%%%%%%%%%%
\begin{frame}[fragile]
  \frametitle{Exercise}
  \Enlarge

  Write a function to sum all of the digits in a number.  %%\pause \emph{I.e.},

  \begin{center}
  \texttt{12145} $\rightarrow$ \texttt{1 + 2 + 1 + 4 + 5} $\rightarrow$ \texttt{13}
  \end{center}
\end{frame}

%%%%%%%%%%%%%%%%%%%%%%%%%%%%%%%%%%%%%%%%%%%%%%%%%%%%%%%%%%%%%%%%%%%%%%%%%%%%%%%%
\begin{frame}[fragile]
  \frametitle{Solution (\texttt{while})}
  \Enlarge

  \begin{semiverbatim}
def sum_digits( n ):
    s = str( n )
    i = 0
    result = 0
    while i < len( s ):
        result = result + int( s[i] )
        i = i + 1
    return result
  \end{semiverbatim}
\end{frame}

%%%%%%%%%%%%%%%%%%%%%%%%%%%%%%%%%%%%%%%%%%%%%%%%%%%%%%%%%%%%%%%%%%%%%%%%%%%%%%%%

%%%%%%%%%%%%%%%%%%%%%%%%%%%%%%%%%%%%%%%%%%%%%%%%%%%%%%%%%%%%%%%%%%%%%%%%%%%%%%%%
\begin{frame}[fragile]
	\frametitle{Example}
	\Enlarge
	
	The following code should increment \texttt{x} if the hundreds place contains a zero:
	\begin{semiverbatim}
        def fun(x): 
           if x < 100 or ???:
              return x+1
           return x 
\end{semiverbatim}
	What should replace the \texttt{???} to complete the code?  Assume \texttt{x} is an integer.
	\begin{enumerate}[label=\Alph*]
		\item  \texttt{x.string(3) == '0'}
		\item  \texttt{str(x)[-3] == '0'}  %$\star$
		\item  \texttt{((x//100) \% 10) == 0}  %$\star$
		\item  None of the above.
	\end{enumerate}
\end{frame}

%%%%%%%%%%%%%%%%%%%%%%%%%%%%%%%%%%%%%%%%

\section{Reminders}

%%%%%%%%%%%%%%%%%%%%%%%%%%%%%%%%%%%%%%%%%%%%%%%%%%%%%%%%%%%%%%%%%%%%%%%%%%%%%%%%
\begin{frame}
  \frametitle{Reminders}
  \Enlarge

  \begin{itemize}
  \myitem  Homework \#2 is due Wed Oct.\ 19 today.
  \myitem  Homework \#3 is due Wed Oct.\ 26.
  \myitem  Midterm \#1 will be on the day of the 12th lecture, covering through Lecture \#11.  
  \end{itemize}
\end{frame}

\end{document}
